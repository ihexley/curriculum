%%%%%%%%%%%%%%%%%
% Date commands %
%%%%%%%%%%%%%%%%%
\newdateformat{experienceDate}{\monthname[\THEMONTH] \THEYEAR}

%% Text used if no "experienceEnd" is specified
\newcommand\currentJobEndDate{%
	\french{actuel}%
	\english{Current}%
}

%% Display dates based on presence and content of experienceStart and experienceEnd
\newcommand\experienceSpan{
	\experienceDate\displaydate{\replacabledatename{experienceStart}} - %
	\ifnum \getdateyear{\replacabledatename{experienceEnd}}=1%
		\currentJobEndDate%
	\else%
		\experienceDate\displaydate{\replacabledatename{experienceEnd}}%
	\fi%
}

%%%%%%%%%%%%%%
% Experience %
%%%%%%%%%%%%%%
%% Define the experience start and end days
%% Format is {day}{month}{year}
%% An experienceEnd set to 1/1/1 represents a current job.
%\newreplacabledate{experienceStart}{1}{10}{2010}
%\newreplacabledate{experienceEnd}{1}{1}{1}

%% CV entry (one for each job and each language)
%% Format is {job description}{company}{City}{Country}
%\english{
%\cventry{\experienceSpan}{Java/JEE/Spring developer/trainer}{Zenika}{Paris}{France}{
%% content
%}
%}

%%%%%%%%%%
% Livres %
%%%%%%%%%%
\newreplacabledate{experienceStart}{8}{6}{2013}
\newreplacabledate{experienceEnd}{1}{1}{1}

\english{
\cventry{\experienceSpan}{Author}{}{}{}{
	\begin{itemize}
		\item \emph{OS X Server : découverte, installation, configuration}: a book about OS X Server, write for beginner to advenced system administrator who wan't to learn about OS X Server in SOHO and SMB. Published by MacGeneration on iBook Store and Kindle.
	\end{itemize}
}
}

\french{
\cventry{\experienceSpan}{Author}{}{}{}{
	\begin{itemize}
		\item \emph{OS X Server : découverte, installation, configuration} : un livre sur OS X Server, fait pour les débutant et les administrateur systèmes avancés souhaitant apprendre à se servir d'OS X Server en TPE et PME. Publié par MacGénération sur l'iBook Store et le Kindle.
	\end{itemize}
}
}

%%%%%%%%%%
% iNig-Services %
%%%%%%%%%%
\newreplacabledate{experienceStart}{2}{1}{2008}
\newreplacabledate{experienceEnd}{1}{1}{1}

\english{
\cventry{\experienceSpan}{Freelance}{iNig-Services}{Paris}{France}{
	\begin{itemize}
		\item Training:
		\begin{itemize}
			\item All Apple’s IT courses. All training end by a certification exam (ACSP, ACTC, ACSA, Xsan 2)
			\item Tech Series ans iOS Technical Training presenter for Apple
			\item OS X and iOS development
		\end{itemize}
		\item Consulting:
		\begin{itemize}
			\item Installation and administration of complete IT assets based on OS X and OS X Server (Directory Server, DNS, DHCP, Mobile Account, Mobile Device Management...)
			\item OS X integration in heterogeneous environment
			\item Development of Mac OS X and iOS applications (Directly for editors like mag-i.fr or visuamobile.com, for in house client usage and for my usage as a system administrator)
			\item Research and Development arround Apple technologie
		\end{itemize}
		\item Blogging:
		\begin{itemize}
			\item Maintainer of blog.inig-service.com about advanced administration of OS X Server
			\item Detailed articles on undocumented features such as "Wide Area Bonjour" (DNS-SD), Password Server password logging...
		\end{itemize}
	\end{itemize}
}
}

\french{
\cventry{\experienceSpan}{Freelance}{iNig-Services}{Paris}{France}{
	\begin{itemize}
		\item Formation:
		\begin{itemize}
			\item Tout le cursus IT d'Apple. Toutes les formations se terminent par une certification (ACSP, ACTC, ACSA, Xsan 2)
			\item Présentateur de Tech Series et iOS Technical Training pour Apple
			\item Développement OS X et iOS
		\end{itemize}
		\item Consulting
		\begin{itemize}
			\item Installation et administration de système d'informations complètement basé sur OS X et OS X Server (avec service d'annuaire, DNS, DHCP, comptes mobiles, MDM…)
			\item Intégration d'OS X en environnement homogène
			\item Développement OS X et iOS (pour des éditeur comme mag-i.fr ou visuamobile.com, pour des clients finaux dans le cas d'application métier, pour les administrateur système)
			\item Recherche et développement autour des technologies Apple
		\end{itemize}
		\item Blogging:
		\begin{itemize}
			\item Auteur de blog.inig-service.com sur l'administration avancée d'OS X Server
			\item Articles détaillés sur des fonction non documenté comme "Wide Area Bonjour" (DNS-SD), l'extraction de mot de passe du Password Server...
		\end{itemize}
	\end{itemize}
}
}


%%%%%%%%%%%
% Centrale Paris %
%%%%%%%%%%%
\newreplacabledate{experienceStart}{1}{2}{2010}
\newreplacabledate{experienceEnd}{1}{1}{1}

\english{
\cventry{\experienceSpan}{Objective-C professor}{École Centrale Paris}{}{France}{
	Training student in System Engineering (Master degree) to iOS and OS X development
	\begin{itemize}
		\item Discovery of the Apple development environment (Xcode suite)
		\item Introduction to Object Oriented Programming
		\item Extensive usage of the Objective-C language and the Cocoa framework
		\item Development of iOS applications
	\end{itemize}
}
}

\french{
\cventry{\experienceSpan}{Vacataire}{École Centrale Paris}{}{France}{
	Formation à un Master en ingénierie système sur le développement iOS et OS X
	\begin{itemize}
		\item Découverte de l'environnement de développement Apple
		\item Introduction au développement objet
		\item Usage poussé du langage Objective-C et du framework Cocoa
		\item Développement d'application iOS
	\end{itemize}
}
}

%%%%%%%%%%%%
% CampusID %
%%%%%%%%%%%%
\newreplacabledate{experienceStart}{1}{9}{2011}
\newreplacabledate{experienceEnd}{1}{6}{2013}

\english{
\cventry{\experienceSpan}{Master trainer on Apple technologies}{Campus ID}{Sophia Antipolis}{France}{
	Teaching iOS and OS X Development.
	\begin{itemize}
		\item Main trainer and author of the Apple courses for bachelor and master
	\end{itemize}
}
}


\french{ 
\cventry{\experienceSpan}{Chargé de cours Apple}{Campus ID}{Sophia Antipolis}{France}{
	Enseignement du développement iOS et OS X.
	\begin{itemize}
		\item Chargé de cours Apple pour les 5 années de cours post bac
	\end{itemize}
}
}

%%%%%%%%%%%
% Supinfo %
%%%%%%%%%%%
\newreplacabledate{experienceStart}{1}{9}{2007}
\newreplacabledate{experienceEnd}{31}{8}{2009}

\english{
\cventry{\experienceSpan}{Supinfo Certified Trainer}{SUPINFO - The International Institute of Information Technology}{}{France}{
	Teaching Mac OS X system administration, iOS and OS X Development.
	\begin{itemize}
		\item Teaching for different Supinfo's campus across the world (France, Morocco, Canada)
		\item Managing students teams working on projects for labo-apple.com
	\end{itemize}
}
}


\french{ 
\cventry{\experienceSpan}{Supinfo Certified Trainer}{SUPINFO - The International Institute of Information Technology}{}{France}{
	Enseignement de l'administration système OS X et du développement iOS et OS X.
	\begin{itemize}
		\item Enseignement pour différents campus Supinfo à travers le monde (En France, au Maroc et au Canada)
		\item Gestion des équipes d'élèves travaillant sur des projets pour labo-apple.com
	\end{itemize}
}
}
