%%%%%%%%%%%%%%%%%
% Date commands %
%%%%%%%%%%%%%%%%%
\newdateformat{experienceDate}{\monthname[\THEMONTH] \THEYEAR}

%% Text used if no "experienceEnd" is specified
\newcommand\currentJobEndDate{%
	\french{actuel}%
	\english{Current}%
}

%% Display dates based on presence and content of experienceStart and experienceEnd
\newcommand\experienceSpan{
	\experienceDate\displaydate{\replacabledatename{experienceStart}} - %
	\ifnum \getdateyear{\replacabledatename{experienceEnd}}=1%
		\currentJobEndDate%
	\else%
		\experienceDate\displaydate{\replacabledatename{experienceEnd}}%
	\fi%
}

%%%%%%%%%%%%%%
% Experience %
%%%%%%%%%%%%%%
%% Define the experience start and end days
%% Format is {day}{month}{year}
%% An experienceEnd set to 1/1/1 represents a current job.
%\newreplacabledate{experienceStart}{1}{10}{2010}
%\newreplacabledate{experienceEnd}{1}{1}{1}

%% CV entry (one for each job and each language)
%% Format is {job description}{company}{City}{Country}
%\english{
%\cventry{\experienceSpan}{Java/JEE/Spring developer/trainer}{Zenika}{Paris}{France}{
%% content
%}
%}

%%%%%%%%%%
% Livres %
%%%%%%%%%%
\newreplacabledate{experienceStart}{8}{6}{2013}
\newreplacabledate{experienceEnd}{1}{1}{1}

\english{
\cventry{\experienceSpan}{Author}{}{}{}{
	\begin{itemize}
		\item \emph{OS X Server : découverte, installation, configuration}. Published by MacGeneration on iBook Store and Kindle.
	\end{itemize}
}
}

\french{
\cventry{\experienceSpan}{Auteur}{}{}{}{
	\begin{itemize}
		\item \emph{OS X Server : découverte, installation, configuration}. Publié par MacGénération sur l'iBook Store et le Kindle.
	\end{itemize}
}
}

%%%%%%%%%%
% iNig-Services %
%%%%%%%%%%
\newreplacabledate{experienceStart}{2}{1}{2008}
\newreplacabledate{experienceEnd}{1}{1}{1}

\english{
\cventry{\experienceSpan}{Freelance}{iNig-Services}{Paris}{France}{
	\begin{itemize}
		\item Training:
		\begin{itemize}
			\item All Apple’s IT courses (ACSP, ACTC, ACSA, Xsan 2)
			\item Tech Series ans iOS Technical Training presenter
			\item OS X and iOS software development
		\end{itemize}
		\item Consulting:
		\begin{itemize}
			\item Installation and administration of complete IT assets based on OS X and OS X Server
			\item OS X integration in heterogeneous environment
			\item Development of Mac OS X and iOS applications
			\item Research and Development arround Apple technologie
		\end{itemize}
		\item Blogging: Maintainer of blog.inig-service.com about advanced administration of OS X Server
	\end{itemize}
}
}

\french{
\cventry{\experienceSpan}{Freelance}{iNig-Services}{Paris}{France}{
	\begin{itemize}
		\item Formation:
		\begin{itemize}
			\item Tout le cursus IT d'Apple (ACSP, ACTC, ACSA, Xsan 2)
			\item Présentateur de Tech Series et iOS Technical Training pour Apple
			\item Développement OS X et iOS
		\end{itemize}
		\item Consulting
		\begin{itemize}
			\item Installation et administration de système d'informations basé sur OS X et OS X Server
			\item Intégration d'OS X en environnement hétérogène
			\item Développement d'application OS X et iOS
			\item Recherche et développement autour des technologies Apple
		\end{itemize}
		\item Blogging : Auteur de blog.inig-service.com sur l'administration avancée d'OS X Server
	\end{itemize}
}
}


%%%%%%%%%%%
% Centrale Paris %
%%%%%%%%%%%
\newreplacabledate{experienceStart}{1}{2}{2010}
\newreplacabledate{experienceEnd}{1}{1}{1}

\english{
\cventry{\experienceSpan}{Objective-C professor}{École Centrale Paris}{}{France}{
	Training student in System Engineering (Master degree) to iOS and OS X development
	\begin{itemize}
		\item Discovery of the Apple development environment (Xcode suite)
		\item Introduction to Object Oriented Programming
		\item Extensive usage of the Objective-C language and the Cocoa framework
		\item Development of iOS applications
	\end{itemize}
}
}

\french{
\cventry{\experienceSpan}{Vacataire}{École Centrale Paris}{}{France}{
	Formation à un Master en ingénierie système sur le développement iOS et OS X
	\begin{itemize}
		\item Découverte de l'environnement de développement Apple
		\item Introduction au développement objet
		\item Usage poussé du langage Objective-C et du framework Cocoa
		\item Développement d'application iOS
	\end{itemize}
}
}

%%%%%%%%%%%%%
%% CampusID %
%%%%%%%%%%%%%
%\newreplacabledate{experienceStart}{1}{9}{2011}
%\newreplacabledate{experienceEnd}{1}{6}{2013}
%
%\english{
%\cventry{\experienceSpan}{Master trainer on Apple technologies}{Campus ID}{Sophia Antipolis}{France}{
%	Teaching iOS and OS X Development.
%	\begin{itemize}
%		\item Main trainer and author of the Apple courses for bachelor and master
%	\end{itemize}
%}
%}
%
%
%\french{ 
%\cventry{\experienceSpan}{Chargé de cours Apple}{Campus ID}{Sophia Antipolis}{France}{
%	Enseignement du développement iOS et OS X.
%	\begin{itemize}
%		\item Chargé de cours Apple pour les 5 années de cours post bac
%	\end{itemize}
%}
%}
%
%%%%%%%%%%%%
%% Supinfo %
%%%%%%%%%%%%
%\newreplacabledate{experienceStart}{1}{9}{2007}
%\newreplacabledate{experienceEnd}{31}{8}{2009}
%
%\english{
%\cventry{\experienceSpan}{Supinfo Certified Trainer}{SUPINFO - The International Institute of Information Technology}{}{France}{
%	Teaching Mac OS X system administration, iOS and OS X Development.
%	\begin{itemize}
%		\item Teaching for different Supinfo's campus across the world (France, Morocco, Canada)
%		\item Managing students teams working on projects for labo-apple.com
%	\end{itemize}
%}
%}
%
%
%\french{ 
%\cventry{\experienceSpan}{Supinfo Certified Trainer}{SUPINFO - The International Institute of Information Technology}{}{France}{
%	Enseignement de l'administration système OS X et du développement iOS et OS X.
%	\begin{itemize}
%		\item Enseignement pour différents campus Supinfo à travers le monde (En France, au Maroc et au Canada)
%		\item Gestion des équipes d'élèves travaillant sur des projets pour labo-apple.com
%	\end{itemize}
%}
%}

